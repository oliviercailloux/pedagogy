\usepackage{xstring}
\usepackage{pgffor}

\ExplSyntaxOn
\NewDocumentCommand{\extractlast}{O{.}mo}
 {
  \seq_set_split:Nnn \l_stroobants_string_seq { #1 } { #2 }
  \IfNoValueTF { #3 }
   {
    \seq_item:Nn \l_stroobants_string_seq { -1 }
   }
   {
    \tl_set:Nx #3 { \seq_item:Nn \l_stroobants_string_seq { -1 } }
   }
 }
\seq_new:N \l_stroobants_string_seq
\ExplSyntaxOff

\newbool{refAPIFormatTypewriter}
%show at sign if present in input
\newbool{refAPIShowAt}
\newbool{refAPIShowPackage}
\newbool{refAPIShowMethod}
\newbool{refAPIShowParameters}
\pgfkeys{
	/refAPI/.is family,
	/refAPI/.cd,
	display string/.estore in = \refAPIDisplayString,
	formatTypewriter/.is if = refAPIFormatTypewriter,
	showAt/.is if = refAPIShowAt,
	showPackage/.is if = refAPIShowPackage,
	showMethod/.is if = refAPIShowMethod,
	showParameters/.is if = refAPIShowParameters,
	default/.style = {
		display string =,
		formatTypewriter = true,
		showAt = true,
		showPackage = false,
		showMethod = true,
		showParameters = false,
	},
}

%Thx https://tex.stackexchange.com/a/34318
%TODO URL without module?
%TODO define the following refAPI macros. Module (before unique / or empty); Method (after unique # or empty); Field (after unique # and before unique ( or empty if no # or after unique # if no (); FQName (between / and # or after / or before # or everything); FQNameSlashes (FQName with slashes instead of dots); Class (between last dot and #).
%OLD - Defines: \refAPIFQName (before #) ; \refAPIFieldFull (after #) ; \refAPIFieldShort (after # and before -) ; \refAPIPackageSlashes (FQName before / with slashes instead of dots) ; \refAPIClass (FQName after /)
%{java.base/java.time.zone.ZoneOffsetTransition} ⇒ https://docs.oracle.com/en/java/javase/11/docs/api/java.base/java/time/zone/ZoneOffsetTransition.html
%{java.base/java.time.zone.ZoneOffsetTransition\#getDateTimeAfter()} ⇒ https://docs.oracle.com/en/java/javase/11/docs/api/java.base/java/time/zone/ZoneOffsetTransition.html#getDateTimeAfter()
%{java.base/java.time.zone.ZoneOffsetTransition\#compareTo(java.time.zone.ZoneOffsetTransition)} ⇒ https://docs.oracle.com/en/java/javase/11/docs/api/java.base/java/time/zone/ZoneOffsetTransition.html#compareTo(java.time.zone.ZoneOffsetTransition)
%{java.base/java.lang.String\#substring(int, int)} ⇒ https://docs.oracle.com/en/java/javase/11/docs/api/java.base/java/lang/String.html#substring(int,int) NB need to drop space in parameter
%{@java.base/java.lang.annotation.Documented} ⇒ https://docs.oracle.com/en/java/javase/11/docs/api/java.base/java/lang/annotation/Documented.html
%{@java.base/java.lang.SuppressWarnings\#value()} ⇒ https://docs.oracle.com/en/java/javase/11/docs/api/java.base/java/lang/SuppressWarnings.html#value()
%The main argument specifies everything needed to build the URL. From this, a default appearance is built, which can be changed with keys. Must be able to: hide @ sign; hide method; hide parameters; hide package; replace everything with provided display string; display in texttt mode or not?.
\newcommand{\jrefAPI}[2][]{%
	\pgfkeys{/refAPI/.cd, default, #1}%
	\IfSubStr{#2}{\#}{%
		\StrBefore{#2}{\#}[\refAPIFQName]%
		\StrBehind{#2}{\#}[\refAPIFieldFull]%
	}{%
		\edef\refAPIFQName{#2}%
		\edef\refAPIFieldFull{}%
	}%
	\IfSubStr{\refAPIFieldFull}{-}{%
		\StrBefore{\refAPIFieldFull}{-}[\refAPIFieldShort]%
	}{%
		\edef\refAPIFieldShort{\refAPIFieldFull}%
	}%
	\StrBefore{\refAPIFQName}{/}[\refAPIPackage]%
	\StrBehind{\refAPIFQName}{/}[\refAPIClass]%
	\StrSubstitute{\refAPIPackage}{.}{/}[\refAPIPackageSlashes]%
	\ifbool{refAPIIsAnnotation}{%
		\edef\refAPIAnnot{@}%
	}{%
		\edef\refAPIAnnot{}%
	}%
	\IfEq{\refAPIFieldFull}{}{%
		{%
			\ifbool{refAPIDisplayFull}{%
				\refAPIPackage%
			}{%
			}%
			\refAPIPrefix{%
				\href{%
					\refAPIBaseUrl\refAPIPackageSlashes/\refAPIClass.html%
				}{%
					\refAPIAnnot\refAPIClass%
				}%
			}%
			\refAPISuffix%
		}%
	}{%
		{%
			\refAPIPrefix{%
				\ifbool{refAPIOmitFQName}{%
				}{%
					\refAPIAnnot\refAPIClass.%
				}%
				\href{%
					\refAPIBaseUrl\refAPIPackageSlashes/\refAPIClass.html\#\refAPIFieldFull%
				}{%
					\refAPIFieldShort%
				}%
			}%
			\refAPISuffix%
		}%
	}%
	\refAPIPrintAll%
}

\newcommand{\refAPIPrintAll}{%
	\begin{description}
		\item[BaseUrl] \foreach \sfx in {BaseUrl}{\csname refAPI\sfx\endcsname}
		\foreach \sfx in {FQName, FieldFull, FieldShort, Package, Class, PackageSlashes}{\item[\sfx] \csname refAPI\sfx\endcsname}
	\end{description}%
}

%\newcommand{\jeeref}[2][]{\jrefAPI[#1, base url = https://docs.oracle.com/javaee/7/api/]{#2}}
\newcommand{\jseref}[2][]{\jrefAPI[#1, base url = https://docs.oracle.com/javase/8/docs/api/]{#2}}
%Standard HTML RenderKit : https://docs.oracle.com/javaee/7/javaserver-faces-2-2/renderkitdocs/HTML_BASIC/javax.faces.Inputjavax.faces.Text.html
%\newcommand{\jsftag}[2][]{\jrefAPI{base url = https://docs.oracle.com/javaee/7/javaserver-faces-2-2/vdldocs-facelets/, full, prefix=:, #1}}

