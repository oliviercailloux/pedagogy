\RequirePackage[l2tabu, orthodox]{nag}
\documentclass[french]{beamer}
\input{preamble/packages}
\input{preamble/math_basics}
\input{preamble/math_mine}
\input{preamble/redac}
\input{preamble/draw}
%\input{preamble/acronyms}

\setbeamertemplate{headline}[singleline]
\setbeamertemplate{footline}[title]

\title{Java Objet : Expériences en cours}
\subject{Pédagogie}
\keywords{projets, alignement}
\author{Olivier Cailloux}
\institute[LAMSADE]{LAMSADE, Université Paris-Dauphine}
\date{\formatdate{4}{6}{2019}}

\begin{document}
\begin{frame}[plain]
	\tikz[remember picture,overlay]{
		\path (current page.south west) node[anchor=south west, inner sep=0] {
			\includegraphics[height=1cm]{LAMSADE95.jpg}
		};
		\path (current page.south east) node[anchor=south east, inner sep=0] {
			\includegraphics[height=9mm]{dauphine_psl2018.png}
		};
		\path (current page.south) ++ (0, 4em) node[anchor=south, inner sep=0] {
			\scriptsize\textcolor{blue}{\url{https://github.com/oliviercailloux/REPO}}
		};
%thanks to https://stackoverflow.com/questions/2423777/is-it-possible-to-create-a-remote-repo-on-github-from-the-cli-without-opening-br/13366414#13366414
%REPO=…
%curl -u 'oliviercailloux' https://api.github.com/user/repos -d "{\"name\":\"${REPO}\"}"
%git remote add origin git@github.com:oliviercailloux/${REPO}.git
%git push --set-upstream origin master
	}
	\titlepage
\end{frame}
\addtocounter{framenumber}{-1}

\begin{frame}
	\frametitle{Outline}
	\tableofcontents[hideallsubsections, sectionstyle=shaded/show]
\end{frame}

\AtBeginSection{
	\begin{frame}
		\frametitle{Outline}
		\tableofcontents[currentsection, hideallsubsections]
	\end{frame}
}

\section{Cadre}
\subsection{Cours}
\begin{frame}
	\frametitle{Cours}
	\begin{itemize}
		\item Java Objet (et Java Projet), MIDO, L3 Apprentissage
		\begin{itemize}
			\item 17 × 3h
			\item 2016--, 2017--, 2018--
		\end{itemize}
		\item Serveurs Java (Java EE), MIDO, M2 SITN Apprentissage 
		\begin{itemize}
			\item 8 × 3h
			\item 2015--, 2016--, 2017--
		\end{itemize}
		\item Serveurs Java (Java EE), MIDO, M2 IF Classique, option 
		\begin{itemize}
			\item 8 × 3h
			\item 2015--, 2016--, 2017--
		\end{itemize}
	\end{itemize}
\end{frame}

\begin{frame}
	\frametitle{En pratique}
	\begin{itemize}
		\item Appui sur livre ouvert de Eck pour cours (peu exploité) et exercices (très exploité)
		\item Tout sur internet (pas de papier)
		\item Exemples de code, diapos
		\item \href{https://github.com/oliviercailloux/java-course/blob/master/Divers/L3a.adoc}{Page} de suivi des séances avec liens vers les explications par sujet traité
		\item Manque (?) : syllabus cohérent
	\end{itemize}
\end{frame}

\section{Alignement pédagogique}
\begin{frame}
	\frametitle{Objectifs}
	\begin{itemize}
		\item Principes d’ingénierie logicielle : code maintenable plutôt que code fonctionnel
		\item Code en pratique : sur machine
		\item Usage des outils modernes d’aide (compilation, avertissements)
		\item Rentabilité du temps investi à maitriser un outil (t.q. git)
		\item Recherche de documentation : code copié d’une source hasardeuse VS source officielle / récente / crédible
		\item Volonté de les faire travailler entre les séances
	\end{itemize}
\end{frame}

\begin{frame}
	\frametitle{Évaluation}
	Fluctuante selon les années !
	\begin{itemize}
		\item Évaluation projet et contrôle continu (pour suivi des explications en séance)
		\item Projet : contributions en binômes puis individuelles puis retour aux binômes
		\item Avec devoirs maison
		\item Avec évaluation orale en séance
		\item Avec exercices en séances
		\item Avec QCMs
	\end{itemize}
\end{frame}

\section{Projets}
\begin{frame}
	\frametitle{Cadre}
	\begin{itemize}
		\item Groupes 4 à 8
		\item Chaque groupe un projet différent
		\item Projets à objectif utile, publication open source visée
		\item Exemple : bibliothèque gestion de préférences ; gestion de bibliographie collaborative ; gestion des cours de Dauphine…
		\item Énoncés courts (2 pages)
	\end{itemize}
\end{frame}

\begin{frame}
	\frametitle{En pratique}
	\begin{itemize}
		\item Appui important sur GitHub et Pull Requests
		\item \href{https://github.com/13tomoore/J-Confs/pull/5\#issuecomment-495630823}{Exemple}
		\item Année découpée en itérations
		\item Travail par binôme sur tâches lors d’une itération
		\item Binômes tournants
		\item Fusion dans master seulement sur mon accord
		\item Feuille de calcul pour garder trace des contributions par binôme par itération
		\item Étudiants fréquemment en difficultés avec git
	\end{itemize}
\end{frame}

\section{Contrôle Continu}
\begin{frame}
	\frametitle{Devoirs maison}
	Remise via GitHub
	\begin{block}{Devoirs du livre}
		\begin{itemize}
			\item Ils ont les corrections à l’avance (!)
			\item Comptent en tout pour 10\% de la note
			\item Notation binaire (fait / pas fait) ; zéro si en retard
			\item Pris sérieusement (?)
		\end{itemize}
	\end{block}
	\begin{block}{Devoirs hors livre}
		\begin{itemize}
			\item −3/20 par heure de retard
			\item Correction automatique après coup
			\item Note transmise via MyCourse
			\item Correction : éléments en séance ou code complet
		\end{itemize}
	\end{block}
\end{frame}

\begin{frame}
	\frametitle{QCM}
	\begin{itemize}
		\item Deux ou trois par cours
		\item Une minute par question
		\item Avec MyCourse (très insatisfaisant)
		\item Correction automatique par MyCourse, résultats immédiats
		\item Points négatifs
		\item Difficile à formuler et calibrer
		\item En pause en attendant meilleurs outils
	\end{itemize}
	NB : également utilisation de WooClap en séance, très appréciée, sans notation
\end{frame}

\begin{frame}
	\frametitle{Exercices notés en séance}
	\begin{itemize}
		\item GitHub Class room
		\item Lien fourni en début de séance (\href{https://classroom.github.com/a/Ny5sBwFU}{exemple})
		\item Chaque étudiant démarre avec une copie du dépôt que j’ai créé au préalable
		\item Énoncé fourni sur le dépôt lui-même
		\item Doivent ajouter du code, des tests, …
		\item Avec documents et internet
		\item 20 à 40 minutes, puis −3/20 par minute de retard
		\item Correction automatique après coup
		\item Outil \emph{presque} formidable
	\end{itemize}
\end{frame}

\section{Bilan}
\begin{frame}
	\frametitle{Quelques bénéfices}
	\begin{itemize}
		\item Commentaires personnalisés sur projets appréciés
		\item Temps important passé en dehors des séances
		\item Compréhension du point de vue de l’ingénierie logicielle (pas seulement code fonctionnel)
		\item Compréhension (trop tardive) de l’intérêt des explications en séances
		\item Capacité à coder
		\item Gestion de projet, sens des reponsabilités
		\item Projets réellement réutilisables au fil du temps
	\end{itemize}
\end{frame}

\begin{frame}
	\frametitle{Quelques faiblesses}
	\begin{itemize}
		\item Refus d’investir du temps pour maitriser ses outils
		\item Recours encore fréquent sans critique à des sources aléatoires
		\item Connaissance intuitive mais pas théorique : faible connaissance des mécanismes de Java, de la théorie des langages, des raisons des choix techniques et des alternatives
		\item Possiblité toujours existante de se cacher derrière les collègues
		\item Difficulté persistante d’évaluer les compétences des étudiants
		\item Difficulté de doser la vitesse d’avancement des séances : fournir assez de connaissances pratiques ; ne pas inonder les étudiants (manque de temps d’exercices selon étudiants)
	\end{itemize}
\end{frame}

\begin{frame}
	\frametitle{Un échec temporaire}
	\begin{itemize}
		\item Des bénéfices en termes pédagogiques
		\item Mais un échec tout de même
	\end{itemize}
	\begin{block}{Analogie boiteuse}
		\begin{itemize}
			\item \og{}J’ai trouvé une méhode pour produire deux fois plus de carottes\fg{}
			\item … mais il faut deux fois plus de travail et de terre
		\end{itemize}
	\end{block}
	\begin{itemize}
		\item La ressource de l’enseignant : le temps (suivi et préparation)
		\item Ici : beaucoup trop important
		\item Espoir d’amélioration future !
	\end{itemize}
\end{frame}

\begin{frame}
	\frametitle{Idées}
	\begin{itemize}
		\item Vidéos pour (re-)accès au contenu nécessaire en fonction du besoin
		\item Démarrage des projets très tôt
		\item Entrainement sur exercices corrigés automatiquement en temps réel
		\item Jointure avec cours d’UML pour séparation ingénierie logicielle et implémentation
	\end{itemize}
\end{frame}

\begin{frame}[plain]
	\addtocounter{framenumber}{-1}
	\begin{center}
		\huge
		\textit{Thank you for your attention!}
	\end{center}
\end{frame}

\end{document}

\appendix
\makeatletter
\def\insertframenumber{\@roman\c@framenumber}
\def\inserttotalframenumber{\@roman\c@framenumber}
\makeatother
\AtBeginSection{
}

\clearpage\pdfbookmark[2]{\refname}{\refname}
\begin{frame}[allowframebreaks]
	\frametitle{\refname}
 	\bibliography{zotero}
\end{frame}

\clearpage\pdfbookmark{License}{License}
\begin{frame}[plain]
	\frametitle{License}
	This presentation, and the associated \LaTeX{} code, are published under the \href{https://opensource.org/licenses/MIT}{MIT license}. Feel free to reuse (parts of) the presentation, under condition that you cite the author.
	
	Credits are to be given to \href{http://www.lamsade.dauphine.fr/~ocailloux/}{Olivier Cailloux}, Université Paris-Dauphine.
\end{frame}
\addtocounter{framenumber}{-1}
\end{document}

\begin{frame}
	\frametitle{Title}
	\begin{block}{Block}
		\begin{itemize}
			\item Item
		\end{itemize}
	\end{block}
	\begin{itemize}
		\item Item
	\end{itemize}
\end{frame}

\begin{frame}
	\frametitle{Title}
	\begin{itemize}
		\item Item
	\end{itemize}
\end{frame}

